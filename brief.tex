\documentclass[a4paper,12pt]{article}
\usepackage[margin=0.8in]{geometry}
\usepackage{titling}
\title{A real time train information and prediction system for the London Underground --- project brief}
\author{Murray Colpman --- Supervisor: Nick Gibbins}
\begin{document}
\maketitle

\section*{Problem}

Although the London Underground has an information system, there is currently
no publicly-available software to track individual trains through the system,
nor is there a way to view the timetable (both planned and predicted) from each
station more than a few trains in advance. This has been done for the National
Rail network, but such a thing has not been attempted for the London
Underground yet.

Such a tracking system would not only be useful for rail enthusiasts, for
example trying to follow a delayed steam service around the network or to
follow a particular train of interest, but would also be very useful for making
a variety of inferences about the network.

The project does not come without its difficulties, however. The main source of
data is from a service called TrackerNet, but this data is cached for around
thirty seconds to prevent flooding. The main accurate source of location is the
signalling block in which the train is located (the track code).  However,
there is no map of where these track codes are in reality. Finally, there
appears to be no common identifier between the timetable data and the live
data.

\section*{Goals and scope}

\begin{enumerate}
  \item To produce a software system to track trains throughout the London
    Underground System and store the results of past days
  \item To infer the working timetable for this data by averaging past runs
  \item To produce a user interface for viewing this data per-station or
    per-train
  \item To expand the system to predict the future arrival and departure times
    of trains, including when forming new services
  \item To produce a topological map of track codes, and to use this map to
    further improve the predictions
\end{enumerate}

The main project comprises goals 1-3. Goals 4 and 5 are stretch goals.
Evaluation of the main project can be performed by testing a sample of the
produced timetable manually against the PDF and seeing how well they match.
Evaluation of goal 4 involves seeing how well the predictions match subsequent
reality. Evaluation of goal 5 is more difficult --- if a portion of a track
code map cannot be obtained, checking the map to ensure that at least the track
layout matches reality would be a reasonable evaluation step.

\end{document}
