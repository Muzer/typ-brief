\documentclass[a4paper,12pt]{article}
\usepackage[margin=1.0in]{geometry}
\title{A real time train information and prediction system for the London Underground --- project brief}
\author{Murray Colpman --- Supervisor: Nick Gibbins}
\begin{document}
\maketitle

\section*{Problem}

Although the London Underground has an information system, there is currently
no publicly-available software to track individual trains through the system,
nor is there a way to view the timetable (both planned and predicted) from each
station more than a few trains in advance. This is possible on the National
Rail network using various sites using Network Rail's open data, but such a
thing has not been attempted for the London Underground yet.

Such a tracking system would not only be useful for rail enthusiasts, for
example trying to follow a delayed steam service around the network or to
follow a particular train of interest, but would also be very useful for making
a variety of inferences about the network.

The project does not come without its difficulties, however. The main source of
data is from a service called TrackerNet, but this data is cached for up to
thirty seconds in the spec (in practice a little longer) to prevent flooding
the (internally-important) system. The main accurate source of location in this
is the logical signalling block in which the train is located (called a track
code on the London Underground). However, there is no map of where these track
codes are in reality, and Transport for London have proven uncooperative in the
task of providing one. Finally, there is, as far as I can tell from preliminary
investigations, no common identifier between the timetable data and the live
data.

\section*{Goals}

\begin{itemize}
  \item To produce a software system to track trains throughout the London
    Underground System and store the results of past days
  \item To infer the working timetable for this data by averaging past runs and
    manually test a sample of this for accuracy against a PDF working timetable
  \item To produce a user interface for viewing this data per-station or
    per-train
  \item To expand the system to predict the future arrival and departure times
    of trains, including when forming new services
  \item To produce a map of track codes
\end{itemize}

\end{document}
